\documentclass[]{article}
\usepackage{lmodern}
\usepackage{amssymb,amsmath}
\usepackage{ifxetex,ifluatex}
\usepackage{fixltx2e} % provides \textsubscript
\ifnum 0\ifxetex 1\fi\ifluatex 1\fi=0 % if pdftex
  \usepackage[T1]{fontenc}
  \usepackage[utf8]{inputenc}
\else % if luatex or xelatex
  \ifxetex
    \usepackage{mathspec}
  \else
    \usepackage{fontspec}
  \fi
  \defaultfontfeatures{Ligatures=TeX,Scale=MatchLowercase}
\fi
% use upquote if available, for straight quotes in verbatim environments
\IfFileExists{upquote.sty}{\usepackage{upquote}}{}
% use microtype if available
\IfFileExists{microtype.sty}{%
\usepackage{microtype}
\UseMicrotypeSet[protrusion]{basicmath} % disable protrusion for tt fonts
}{}
\usepackage[margin=1in]{geometry}
\usepackage{hyperref}
\hypersetup{unicode=true,
            pdftitle={FE8828 Programming Web Applications in Finance},
            pdfauthor={Dr.~Yang Ye  \textless{}Email:yy@runchee.com\textgreater{}},
            pdfborder={0 0 0},
            breaklinks=true}
\urlstyle{same}  % don't use monospace font for urls
\usepackage{graphicx,grffile}
\makeatletter
\def\maxwidth{\ifdim\Gin@nat@width>\linewidth\linewidth\else\Gin@nat@width\fi}
\def\maxheight{\ifdim\Gin@nat@height>\textheight\textheight\else\Gin@nat@height\fi}
\makeatother
% Scale images if necessary, so that they will not overflow the page
% margins by default, and it is still possible to overwrite the defaults
% using explicit options in \includegraphics[width, height, ...]{}
\setkeys{Gin}{width=\maxwidth,height=\maxheight,keepaspectratio}
\usepackage[normalem]{ulem}
% avoid problems with \sout in headers with hyperref:
\pdfstringdefDisableCommands{\renewcommand{\sout}{}}
\IfFileExists{parskip.sty}{%
\usepackage{parskip}
}{% else
\setlength{\parindent}{0pt}
\setlength{\parskip}{6pt plus 2pt minus 1pt}
}
\setlength{\emergencystretch}{3em}  % prevent overfull lines
\providecommand{\tightlist}{%
  \setlength{\itemsep}{0pt}\setlength{\parskip}{0pt}}
\setcounter{secnumdepth}{0}
% Redefines (sub)paragraphs to behave more like sections
\ifx\paragraph\undefined\else
\let\oldparagraph\paragraph
\renewcommand{\paragraph}[1]{\oldparagraph{#1}\mbox{}}
\fi
\ifx\subparagraph\undefined\else
\let\oldsubparagraph\subparagraph
\renewcommand{\subparagraph}[1]{\oldsubparagraph{#1}\mbox{}}
\fi

%%% Use protect on footnotes to avoid problems with footnotes in titles
\let\rmarkdownfootnote\footnote%
\def\footnote{\protect\rmarkdownfootnote}

%%% Change title format to be more compact
\usepackage{titling}

% Create subtitle command for use in maketitle
\newcommand{\subtitle}[1]{
  \posttitle{
    \begin{center}\large#1\end{center}
    }
}

\setlength{\droptitle}{-2em}

  \title{FE8828 Programming Web Applications in Finance}
    \pretitle{\vspace{\droptitle}\centering\huge}
  \posttitle{\par}
  \subtitle{Final Assignment - Supplementary material}
  \author{Dr.~Yang Ye
\textless{}Email:\href{mailto:yy@runchee.com}{\nolinkurl{yy@runchee.com}}\textgreater{}}
    \preauthor{\centering\large\emph}
  \postauthor{\par}
      \predate{\centering\large\emph}
  \postdate{\par}
    \date{Nov 26, 2018}


\begin{document}
\maketitle

\begin{enumerate}
\def\labelenumi{\arabic{enumi}.}
\tightlist
\item
  Bank management
\end{enumerate}

\begin{itemize}
\item
  Design of the data frames

\begin{verbatim}
Data frame 1: Account 
| AcountNo | Name | Credit |

Data frame 2: Transaction
| TransactionNo | Date | AccountNo | TransactionType | Amount | Currency |

Data frame 3: Currency to SGD
| Currency | Conversion | Date |
\end{verbatim}
\item
  There are three kinds of TransctionType: Deposit/Withdraw/Spend.
  Amount is of sign +/-/- respectively.
\item
  Deposit / Withdraw is paired up. You can't withdraw more than deposit.
\item
  Credit / Spend is paired up. You can't spend more than credit.
  (Simplified than earlier version of \sout{spendt \textless{}= credit +
  deposit})
\item
  Assume credit lasts from 2017-07-01 and 2017-09-30. The monthly
  repayment of credit is out of scope of this project.
\item
  Example Sections in the solution

  \begin{itemize}
  \tightlist
  \item
    Generate test data in three data frames.
  \item
    Pick one random AcountNo to show the monthly statement.
  \item
    Pick a random date between 2017-07-01 and 2017-09-30 to do

    \begin{itemize}
    \tightlist
    \item
      Risk department:

      \begin{itemize}
      \tightlist
      \item
        Total balance daily
      \item
        Total receivable from credit daily
      \item
        Top 10 High and low-risk client (i.e.~balance - spent ).
      \end{itemize}
    \end{itemize}
  \item
    Show what's by the end of the period, on the end of day of
    2017-09-30.

    \begin{itemize}
    \tightlist
    \item
      Customer department:

      \begin{itemize}
      \tightlist
      \item
        Top 10 customer with large balance (deposit - withdraw)
      \item
        Top 10 spending customer (spend)
      \item
        Top 10 saving customer (deposit)
      \end{itemize}
    \item
      Treasury department:

      \begin{itemize}
      \tightlist
      \item
        Interest that all customers need to pay for three months
      \item
        Assume annual interest is 0.25\%
      \item
        Interest starts when customer spends.
      \end{itemize}
    \end{itemize}
  \end{itemize}
\end{itemize}

\begin{enumerate}
\def\labelenumi{\arabic{enumi}.}
\setcounter{enumi}{1}
\tightlist
\item
  Delta Hedging
\end{enumerate}

Show case how delta hedging works as a trading strategy.

\begin{itemize}
\tightlist
\item
  Show how one trade works
\item
  Backtest for available history.
\end{itemize}

Assumption:

\begin{itemize}
\item
  You hold 100 ATM call/put option which expires in 30 days (calendar
  days). You just need to do either Call or Put.
\item
  You start to do delta hedging daily immediately till 2nd last day. You
  close stock position in the last day.
\item
  Delta hedging: calculate the delta from option, negate it, that's the
  quantity what you need to hold over 1 day. Repeat for every trading
  day.
\item
  Daily PnL: (option premium change) + (stock holding quantity * price
  change).
\item
  You can get your favorite stocks here. There is one year of data.

  \begin{itemize}
  \tightlist
  \item
    \textless{}\url{https://marketchameleon.com/Overview/\%7BStock}
    Code\}/DailyHistory/\textgreater{}
  \item
    e.g.: \url{https://marketchameleon.com/Overview/GS/DailyHistory/}
  \end{itemize}
\item
  Daily IV30 is provided.
\item
  As underlying is equity, dividend yield is applicable for B-S
  valuation.
\item
  US risk-free rate for 1M: 0.8\% (annualized)
\item
  Create xts object from the data from website.
\item
  One trade analysis

  \begin{itemize}
  \item
    Pick a date range using xts object.
  \item
    Get starting date and end date.

\begin{verbatim}
  dates <- index(xts_obj)
  start_date <- min(dates)
  end_date <- max(dates)
  start_price <- xts_obj[start_date, "Close"]
  start_volatility <- xts_obj[start_date, "IV30"]
\end{verbatim}
  \item
    create a df with date column

\begin{verbatim}
  df <- tibble(date = dates)
  df$Close <- coredata(xts_obj[, "Close"])
\end{verbatim}
  \item
    Daily Profit and Loss (``DoD PnL'')

    \begin{itemize}
    \item
      Option side:

\begin{verbatim}
X <- start_price
sigma = start_volatility
r <- 0.8 / 100
# Vary S and Time everyday
S <- Close
Time <- (end_date - date) / 365
GBSOption(TypeFlag, S, X, Time, r, b, sigma)@price

df_opt <- rowwise(df) %>%
          mutate(premium = GBSOption(TypeFlag = "...",
                                     S = Close,
                                     X = start_price,
                                     Time = (end_date - date) / 365,
                                     r = ..., # interest rate
                                     b = ..., # dividend yield
                                     sigma = start_volatility)@price) %>%
          ungroup %>%
          mutate(Option_DoD_PnL = ifelse(date == start_date,
          # On the 1st date, we count the cost of buying the option
                                         premium * (-1), 
                                         premium - lag(premium)))
\end{verbatim}
    \item
      Hedging side:

\begin{verbatim}
rowwise() %>%
mutate(delta_hedge = GBSGreeks("delta", TypeFlag, S, X, Time, r, b, sigma) *
                     quantity * (-1)) %>%
ungroup() %>%
mutate(Hedging_DoD_Pnl = ifelse(date == start_date,
                                0,
                                delta_hedge * (Close - lag(Close))))
\end{verbatim}
    \item
      Daily PnL (combined):

\begin{verbatim}
mutate(DoD_PnL = Option_DoD_PnL + Hedging_DoD_Pnl)
\end{verbatim}
    \end{itemize}
  \end{itemize}
\item
  Max Drawdown: accumulative of Daily PnL, max - min.

\begin{verbatim}
ungroup() %>%
mutate(PnL = cumsum(DoD_PnL)) %>%
{
  xs <- .$PnL
  max(cummax(xs) - cummin(xs))
}
\end{verbatim}
\end{itemize}


\end{document}
