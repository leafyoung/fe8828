\documentclass[]{article}
\usepackage{lmodern}
\usepackage{amssymb,amsmath}
\usepackage{ifxetex,ifluatex}
\usepackage{fixltx2e} % provides \textsubscript
\ifnum 0\ifxetex 1\fi\ifluatex 1\fi=0 % if pdftex
  \usepackage[T1]{fontenc}
  \usepackage[utf8]{inputenc}
\else % if luatex or xelatex
  \ifxetex
    \usepackage{mathspec}
  \else
    \usepackage{fontspec}
  \fi
  \defaultfontfeatures{Ligatures=TeX,Scale=MatchLowercase}
\fi
% use upquote if available, for straight quotes in verbatim environments
\IfFileExists{upquote.sty}{\usepackage{upquote}}{}
% use microtype if available
\IfFileExists{microtype.sty}{%
\usepackage{microtype}
\UseMicrotypeSet[protrusion]{basicmath} % disable protrusion for tt fonts
}{}
\usepackage[margin=1in]{geometry}
\usepackage{hyperref}
\hypersetup{unicode=true,
            pdftitle={R Markdown assignment 4\#},
            pdfauthor={Abhishek Vijaykumar},
            pdfborder={0 0 0},
            breaklinks=true}
\urlstyle{same}  % don't use monospace font for urls
\usepackage{color}
\usepackage{fancyvrb}
\newcommand{\VerbBar}{|}
\newcommand{\VERB}{\Verb[commandchars=\\\{\}]}
\DefineVerbatimEnvironment{Highlighting}{Verbatim}{commandchars=\\\{\}}
% Add ',fontsize=\small' for more characters per line
\usepackage{framed}
\definecolor{shadecolor}{RGB}{248,248,248}
\newenvironment{Shaded}{\begin{snugshade}}{\end{snugshade}}
\newcommand{\KeywordTok}[1]{\textcolor[rgb]{0.13,0.29,0.53}{\textbf{#1}}}
\newcommand{\DataTypeTok}[1]{\textcolor[rgb]{0.13,0.29,0.53}{#1}}
\newcommand{\DecValTok}[1]{\textcolor[rgb]{0.00,0.00,0.81}{#1}}
\newcommand{\BaseNTok}[1]{\textcolor[rgb]{0.00,0.00,0.81}{#1}}
\newcommand{\FloatTok}[1]{\textcolor[rgb]{0.00,0.00,0.81}{#1}}
\newcommand{\ConstantTok}[1]{\textcolor[rgb]{0.00,0.00,0.00}{#1}}
\newcommand{\CharTok}[1]{\textcolor[rgb]{0.31,0.60,0.02}{#1}}
\newcommand{\SpecialCharTok}[1]{\textcolor[rgb]{0.00,0.00,0.00}{#1}}
\newcommand{\StringTok}[1]{\textcolor[rgb]{0.31,0.60,0.02}{#1}}
\newcommand{\VerbatimStringTok}[1]{\textcolor[rgb]{0.31,0.60,0.02}{#1}}
\newcommand{\SpecialStringTok}[1]{\textcolor[rgb]{0.31,0.60,0.02}{#1}}
\newcommand{\ImportTok}[1]{#1}
\newcommand{\CommentTok}[1]{\textcolor[rgb]{0.56,0.35,0.01}{\textit{#1}}}
\newcommand{\DocumentationTok}[1]{\textcolor[rgb]{0.56,0.35,0.01}{\textbf{\textit{#1}}}}
\newcommand{\AnnotationTok}[1]{\textcolor[rgb]{0.56,0.35,0.01}{\textbf{\textit{#1}}}}
\newcommand{\CommentVarTok}[1]{\textcolor[rgb]{0.56,0.35,0.01}{\textbf{\textit{#1}}}}
\newcommand{\OtherTok}[1]{\textcolor[rgb]{0.56,0.35,0.01}{#1}}
\newcommand{\FunctionTok}[1]{\textcolor[rgb]{0.00,0.00,0.00}{#1}}
\newcommand{\VariableTok}[1]{\textcolor[rgb]{0.00,0.00,0.00}{#1}}
\newcommand{\ControlFlowTok}[1]{\textcolor[rgb]{0.13,0.29,0.53}{\textbf{#1}}}
\newcommand{\OperatorTok}[1]{\textcolor[rgb]{0.81,0.36,0.00}{\textbf{#1}}}
\newcommand{\BuiltInTok}[1]{#1}
\newcommand{\ExtensionTok}[1]{#1}
\newcommand{\PreprocessorTok}[1]{\textcolor[rgb]{0.56,0.35,0.01}{\textit{#1}}}
\newcommand{\AttributeTok}[1]{\textcolor[rgb]{0.77,0.63,0.00}{#1}}
\newcommand{\RegionMarkerTok}[1]{#1}
\newcommand{\InformationTok}[1]{\textcolor[rgb]{0.56,0.35,0.01}{\textbf{\textit{#1}}}}
\newcommand{\WarningTok}[1]{\textcolor[rgb]{0.56,0.35,0.01}{\textbf{\textit{#1}}}}
\newcommand{\AlertTok}[1]{\textcolor[rgb]{0.94,0.16,0.16}{#1}}
\newcommand{\ErrorTok}[1]{\textcolor[rgb]{0.64,0.00,0.00}{\textbf{#1}}}
\newcommand{\NormalTok}[1]{#1}
\usepackage{graphicx,grffile}
\makeatletter
\def\maxwidth{\ifdim\Gin@nat@width>\linewidth\linewidth\else\Gin@nat@width\fi}
\def\maxheight{\ifdim\Gin@nat@height>\textheight\textheight\else\Gin@nat@height\fi}
\makeatother
% Scale images if necessary, so that they will not overflow the page
% margins by default, and it is still possible to overwrite the defaults
% using explicit options in \includegraphics[width, height, ...]{}
\setkeys{Gin}{width=\maxwidth,height=\maxheight,keepaspectratio}
\IfFileExists{parskip.sty}{%
\usepackage{parskip}
}{% else
\setlength{\parindent}{0pt}
\setlength{\parskip}{6pt plus 2pt minus 1pt}
}
\setlength{\emergencystretch}{3em}  % prevent overfull lines
\providecommand{\tightlist}{%
  \setlength{\itemsep}{0pt}\setlength{\parskip}{0pt}}
\setcounter{secnumdepth}{0}
% Redefines (sub)paragraphs to behave more like sections
\ifx\paragraph\undefined\else
\let\oldparagraph\paragraph
\renewcommand{\paragraph}[1]{\oldparagraph{#1}\mbox{}}
\fi
\ifx\subparagraph\undefined\else
\let\oldsubparagraph\subparagraph
\renewcommand{\subparagraph}[1]{\oldsubparagraph{#1}\mbox{}}
\fi

%%% Use protect on footnotes to avoid problems with footnotes in titles
\let\rmarkdownfootnote\footnote%
\def\footnote{\protect\rmarkdownfootnote}

%%% Change title format to be more compact
\usepackage{titling}

% Create subtitle command for use in maketitle
\newcommand{\subtitle}[1]{
  \posttitle{
    \begin{center}\large#1\end{center}
    }
}

\setlength{\droptitle}{-2em}

  \title{R Markdown assignment 4\#}
    \pretitle{\vspace{\droptitle}\centering\huge}
  \posttitle{\par}
    \author{Abhishek Vijaykumar}
    \preauthor{\centering\large\emph}
  \postauthor{\par}
      \predate{\centering\large\emph}
  \postdate{\par}
    \date{December 3rd, 2018}

\usepackage{booktabs}
\usepackage{longtable}
\usepackage{array}
\usepackage{multirow}
\usepackage[table]{xcolor}
\usepackage{wrapfig}
\usepackage{float}
\usepackage{colortbl}
\usepackage{pdflscape}
\usepackage{tabu}
\usepackage{threeparttable}
\usepackage{threeparttablex}
\usepackage[normalem]{ulem}
\usepackage{makecell}

\usepackage{leading}
\leading{18pt}

\begin{document}
\maketitle

\subsection{R Markdown}\label{r-markdown}

\subsubsection{Here are the ten insights into the Banking Data using
GGPLOT}\label{here-are-the-ten-insights-into-the-banking-data-using-ggplot}

\subsection{Including Plots}\label{including-plots}

\subsubsection{1) Distribution of contact method across
age}\label{distribution-of-contact-method-across-age}

\paragraph{Observation - Mostly people below the age of 60 have
celephones, the older generations mostly still use
telephones}\label{observation---mostly-people-below-the-age-of-60-have-celephones-the-older-generations-mostly-still-use-telephones}

\begin{Shaded}
\begin{Highlighting}[]
\KeywordTok{ggplot}\NormalTok{(bank) }\OperatorTok{+}
\KeywordTok{geom_bar}\NormalTok{(}\DataTypeTok{mapping =} \KeywordTok{aes}\NormalTok{(}\DataTypeTok{x =}\NormalTok{ age,}\DataTypeTok{fill=}\NormalTok{contact)) }\OperatorTok{+}\KeywordTok{theme}\NormalTok{(}\DataTypeTok{axis.text.x =} \KeywordTok{element_text}\NormalTok{(}\DataTypeTok{angle =} \DecValTok{90}\NormalTok{, }\DataTypeTok{hjust =} \DecValTok{1}\NormalTok{))}\OperatorTok{+}\KeywordTok{labs}\NormalTok{(}\DataTypeTok{title =} \StringTok{"Distribution of contact with age"}\NormalTok{,}
\DataTypeTok{caption =} \StringTok{"Bank Data"}\NormalTok{) }\OperatorTok{+}
\KeywordTok{labs}\NormalTok{(}\DataTypeTok{x =} \StringTok{"age"}\NormalTok{, }\DataTypeTok{y =} \StringTok{"Count"}\NormalTok{)}
\end{Highlighting}
\end{Shaded}

\includegraphics{R_Markdown_Assignment__4_files/figure-latex/Query 1-1.pdf}

\begin{Shaded}
\begin{Highlighting}[]
\KeywordTok{ggplot}\NormalTok{(bank) }\OperatorTok{+}
\KeywordTok{geom_bar}\NormalTok{(}\DataTypeTok{mapping =} \KeywordTok{aes}\NormalTok{(}\DataTypeTok{x =}\NormalTok{ age,}\DataTypeTok{fill=}\NormalTok{contact),}\DataTypeTok{position=}\StringTok{"fill"}\NormalTok{) }\OperatorTok{+}\KeywordTok{theme}\NormalTok{(}\DataTypeTok{axis.text.x =} \KeywordTok{element_text}\NormalTok{(}\DataTypeTok{angle =} \DecValTok{90}\NormalTok{, }\DataTypeTok{hjust =} \DecValTok{1}\NormalTok{))}\OperatorTok{+}\KeywordTok{labs}\NormalTok{(}\DataTypeTok{title =} \StringTok{"Distribution of contact with age"}\NormalTok{,}\DataTypeTok{x =} \StringTok{"age"}\NormalTok{, }\DataTypeTok{y =} \StringTok{"Count"}\NormalTok{,}
\DataTypeTok{caption =} \StringTok{"Bank Data"}\NormalTok{) }
\end{Highlighting}
\end{Shaded}

\includegraphics{R_Markdown_Assignment__4_files/figure-latex/Query 1-2.pdf}

\subsubsection{2) Average Balance of people based on
profession}\label{average-balance-of-people-based-on-profession}

\paragraph{Observation - Retired people have the highest balance,
believeable, but housemaids have the second highest? The next Query will
try to findout which profession has data points that deviate from the
median}\label{observation---retired-people-have-the-highest-balance-believeable-but-housemaids-have-the-second-highest-the-next-query-will-try-to-findout-which-profession-has-data-points-that-deviate-from-the-median}

\begin{Shaded}
\begin{Highlighting}[]
\KeywordTok{ggplot}\NormalTok{()}\OperatorTok{+}
\StringTok{ }\KeywordTok{geom_point}\NormalTok{(}\DataTypeTok{data =} \KeywordTok{group_by}\NormalTok{(bank, job) }\OperatorTok
\KeywordTok{summarize}\NormalTok{(}\DataTypeTok{avgbal =} \KeywordTok{mean}\NormalTok{(balance)) }\OperatorTok\StringTok{ }\NormalTok{ungroup,}
\KeywordTok{aes}\NormalTok{(}\DataTypeTok{x =}\NormalTok{ job, }\DataTypeTok{y =}\NormalTok{ avgbal), }\DataTypeTok{size =} \DecValTok{4}\NormalTok{, }\DataTypeTok{color=}\StringTok{"blue"}\NormalTok{,}\DataTypeTok{alpha =} \DecValTok{1}\OperatorTok{/}\DecValTok{3}\NormalTok{, }\DataTypeTok{inherit.aes =} \OtherTok{FALSE}\NormalTok{) }\OperatorTok{+}\KeywordTok{theme}\NormalTok{(}\DataTypeTok{axis.text.x =} \KeywordTok{element_text}\NormalTok{(}\DataTypeTok{angle =} \DecValTok{90}\NormalTok{, }\DataTypeTok{hjust =} \DecValTok{1}\NormalTok{))}\OperatorTok{+}\KeywordTok{labs}\NormalTok{(}\DataTypeTok{title =} \StringTok{"Average bank balance by profession"}\NormalTok{,}\DataTypeTok{x =} \StringTok{"job"}\NormalTok{, }\DataTypeTok{y =} \StringTok{"Avg Bank Balance"}\NormalTok{,}
\DataTypeTok{caption =} \StringTok{"Bank Data"}\NormalTok{) }
\end{Highlighting}
\end{Shaded}

\includegraphics{R_Markdown_Assignment__4_files/figure-latex/Query 2-1.pdf}

\subsubsection{3) Looking for Outliers in Bank balance in each
profession by examinimg the boxplot for the
data}\label{looking-for-outliers-in-bank-balance-in-each-profession-by-examinimg-the-boxplot-for-the-data}

\paragraph{\texorpdfstring{The two outliers present under ``retired''
and ``entrepreneur'' make it difficult to look at the box
plot}{The two outliers present under retired and entrepreneur make it difficult to look at the box plot}}\label{the-two-outliers-present-under-retired-and-entrepreneur-make-it-difficult-to-look-at-the-box-plot}

\begin{Shaded}
\begin{Highlighting}[]
\KeywordTok{ggplot}\NormalTok{(bank, }\KeywordTok{aes}\NormalTok{(}\DataTypeTok{x =}\NormalTok{ job, }\DataTypeTok{y =}\NormalTok{ balance))}\OperatorTok{+}\StringTok{ }\KeywordTok{geom_boxplot}\NormalTok{() }\OperatorTok{+}\StringTok{ }\KeywordTok{theme}\NormalTok{(}\DataTypeTok{axis.text.x =} \KeywordTok{element_text}\NormalTok{(}\DataTypeTok{angle =} \DecValTok{90}\NormalTok{, }\DataTypeTok{hjust =} \DecValTok{1}\NormalTok{))}\OperatorTok{+}\KeywordTok{labs}\NormalTok{(}\DataTypeTok{title =} \StringTok{"BoxPlot for average bank balance by profession"}\NormalTok{,}\DataTypeTok{x =} \StringTok{"job"}\NormalTok{, }\DataTypeTok{y =} \StringTok{"Bank Balance"}\NormalTok{,}
\DataTypeTok{caption =} \StringTok{"Bank Data"}\NormalTok{) }
\end{Highlighting}
\end{Shaded}

\includegraphics{R_Markdown_Assignment__4_files/figure-latex/Query 3-1.pdf}

\paragraph{\texorpdfstring{Once these are removed, we are able to
clearly see how the ``housemaid'' and ``retired'' profession category is
being impacted by the presence of
outliers}{Once these are removed, we are able to clearly see how the housemaid and retired profession category is being impacted by the presence of outliers}}\label{once-these-are-removed-we-are-able-to-clearly-see-how-the-housemaid-and-retired-profession-category-is-being-impacted-by-the-presence-of-outliers}

\begin{Shaded}
\begin{Highlighting}[]
\NormalTok{df<-}\KeywordTok{arrange}\NormalTok{(bank,}\KeywordTok{desc}\NormalTok{(balance))}
\NormalTok{df}\OperatorTok{$}\NormalTok{balance[}\DecValTok{1}\OperatorTok{:}\DecValTok{2}\NormalTok{]=}\KeywordTok{mean}\NormalTok{(df}\OperatorTok{$}\NormalTok{balance)}
\KeywordTok{ggplot}\NormalTok{(df, }\KeywordTok{aes}\NormalTok{(}\DataTypeTok{x =}\NormalTok{ job, }\DataTypeTok{y =}\NormalTok{ balance))}\OperatorTok{+}\StringTok{ }\KeywordTok{geom_boxplot}\NormalTok{() }\OperatorTok{+}\StringTok{ }\KeywordTok{theme}\NormalTok{(}\DataTypeTok{axis.text.x =} \KeywordTok{element_text}\NormalTok{(}\DataTypeTok{angle =} \DecValTok{90}\NormalTok{, }\DataTypeTok{hjust =} \DecValTok{1}\NormalTok{))}\OperatorTok{+}\KeywordTok{labs}\NormalTok{(}\DataTypeTok{title =} \StringTok{"BoxPlot for average bank balance by profession"}\NormalTok{,}\DataTypeTok{x =} \StringTok{"job"}\NormalTok{, }\DataTypeTok{y =} \StringTok{"Bank Balance"}\NormalTok{,}
\DataTypeTok{caption =} \StringTok{"Bank Data"}\NormalTok{) }\OperatorTok{+}\KeywordTok{labs}\NormalTok{(}\DataTypeTok{title =} \StringTok{"BoxPlot for average bank balance by profession"}\NormalTok{,}\DataTypeTok{x =} \StringTok{"job"}\NormalTok{, }\DataTypeTok{y =} \StringTok{"Bank Balance"}\NormalTok{,}
\DataTypeTok{caption =} \StringTok{"Bank Data"}\NormalTok{) }
\end{Highlighting}
\end{Shaded}

\includegraphics{R_Markdown_Assignment__4_files/figure-latex/Query a-1.pdf}

\subsubsection{4) Which employment category is most frequently contacted
by bank
employees}\label{which-employment-category-is-most-frequently-contacted-by-bank-employees}

\paragraph{\texorpdfstring{``Pdays'' is the days since the last time the
person was contacted, the more left skewed the distribution, the more
number of times that particular profession group was contacted, from
this we see that entreprenuer's are amongst the most frequently while
unemployed people are amongst the
least}{Pdays is the days since the last time the person was contacted, the more left skewed the distribution, the more number of times that particular profession group was contacted, from this we see that entreprenuer's are amongst the most frequently while unemployed people are amongst the least}}\label{pdays-is-the-days-since-the-last-time-the-person-was-contacted-the-more-left-skewed-the-distribution-the-more-number-of-times-that-particular-profession-group-was-contacted-from-this-we-see-that-entreprenuers-are-amongst-the-most-frequently-while-unemployed-people-are-amongst-the-least}

\begin{Shaded}
\begin{Highlighting}[]
\KeywordTok{ggplot}\NormalTok{(bank, }\KeywordTok{aes}\NormalTok{(pdays)) }\OperatorTok{+}
\KeywordTok{geom_density}\NormalTok{(}\DataTypeTok{data =}\NormalTok{ dplyr}\OperatorTok{::}\KeywordTok{filter}\NormalTok{(bank, pdays }\OperatorTok{>}\DecValTok{0}\NormalTok{),}\KeywordTok{aes}\NormalTok{(}\DataTypeTok{fill =}\NormalTok{ job), }\DataTypeTok{alpha =} \FloatTok{0.5}\NormalTok{) }\OperatorTok{+}\KeywordTok{labs}\NormalTok{(}\DataTypeTok{title =} \StringTok{"Desnity plot for Pdays grouped by profession"}\NormalTok{,}\DataTypeTok{x =} \StringTok{"Pdays"}\NormalTok{, }\DataTypeTok{y =} \StringTok{"Count"}\NormalTok{,}
\DataTypeTok{caption =} \StringTok{"Bank Data"}\NormalTok{) }
\end{Highlighting}
\end{Shaded}

\includegraphics{R_Markdown_Assignment__4_files/figure-latex/Query 4-1.pdf}

\subsubsection{9) Proportion of defaults according to
profession}\label{proportion-of-defaults-according-to-profession}

\paragraph{Here we see that the profession group that defaults the most
is
entrepreneurs}\label{here-we-see-that-the-profession-group-that-defaults-the-most-is-entrepreneurs}

\begin{Shaded}
\begin{Highlighting}[]
\KeywordTok{ggplot}\NormalTok{(}\DataTypeTok{data =}\NormalTok{ bank, }\DataTypeTok{mapping =} \KeywordTok{aes}\NormalTok{(}\DataTypeTok{x =}\NormalTok{ job, }\DataTypeTok{fill =}\NormalTok{ default)) }\OperatorTok{+}
\KeywordTok{geom_bar}\NormalTok{(}\DataTypeTok{position=}\StringTok{"fill"}\NormalTok{)  }\OperatorTok{+}
\KeywordTok{theme}\NormalTok{(}\DataTypeTok{axis.text.x =} \KeywordTok{element_text}\NormalTok{(}\DataTypeTok{angle =} \DecValTok{90}\NormalTok{, }\DataTypeTok{hjust =} \DecValTok{1}\NormalTok{))}\OperatorTok{+}\KeywordTok{coord_flip}\NormalTok{()}\OperatorTok{+}\KeywordTok{labs}\NormalTok{(}\DataTypeTok{title =} \StringTok{"Filled Bar Plot for the default count"}\NormalTok{,}\DataTypeTok{x =} \StringTok{"Pdays"}\NormalTok{, }\DataTypeTok{y =} \StringTok{"Count"}\NormalTok{,}\DataTypeTok{caption =} \StringTok{"Bank Data"}\NormalTok{) }
\end{Highlighting}
\end{Shaded}

\includegraphics{R_Markdown_Assignment__4_files/figure-latex/Query 5-1.pdf}

\subsubsection{6) Which bucket of Call duration has the highest
probability of getting a
customer}\label{which-bucket-of-call-duration-has-the-highest-probability-of-getting-a-customer}

\paragraph{\texorpdfstring{Here we confirm that the longer a person is
on a call, the more likely he is to agree to opening a term deposit
(Column
``y'')}{Here we confirm that the longer a person is on a call, the more likely he is to agree to opening a term deposit (Column y)}}\label{here-we-confirm-that-the-longer-a-person-is-on-a-call-the-more-likely-he-is-to-agree-to-opening-a-term-deposit-column-y}

\begin{Shaded}
\begin{Highlighting}[]
\NormalTok{df<-bank}
\NormalTok{df}\OperatorTok{$}\NormalTok{duration_quantiles <-}\StringTok{ }\KeywordTok{cut}\NormalTok{(df}\OperatorTok{$}\NormalTok{duration, }\KeywordTok{quantile}\NormalTok{(df}\OperatorTok{$}\NormalTok{duration, }\KeywordTok{seq}\NormalTok{(}\DecValTok{0}\NormalTok{, }\DecValTok{1}\NormalTok{, }\DataTypeTok{length =} \DecValTok{15}\NormalTok{), }\DataTypeTok{na.rm =} \OtherTok{TRUE}\NormalTok{))}
\KeywordTok{ggplot}\NormalTok{(df) }\OperatorTok{+}
\StringTok{ }\KeywordTok{geom_point}\NormalTok{(}\KeywordTok{group_by}\NormalTok{(df,duration_quantiles)}\OperatorTok\KeywordTok{summarise}\NormalTok{(}\DataTypeTok{percentyes=}\KeywordTok{sum}\NormalTok{(y }\OperatorTok{==}\StringTok{ "yes"}\NormalTok{) }\OperatorTok{/}\StringTok{ }\KeywordTok{n}\NormalTok{())}\OperatorTok\NormalTok{dplyr}\OperatorTok{::}\KeywordTok{filter}\NormalTok{(duration_quantiles}\OperatorTok{!=}\StringTok{"NA"}\NormalTok{)}\OperatorTok\KeywordTok{arrange}\NormalTok{(percentyes)}\OperatorTok\KeywordTok{arrange}\NormalTok{(duration_quantiles),}\DataTypeTok{mapping=}\KeywordTok{aes}\NormalTok{(}\DataTypeTok{x=}\NormalTok{duration_quantiles,}\DataTypeTok{y=}\NormalTok{percentyes),}\DataTypeTok{alpha =} \DecValTok{1}\OperatorTok{/}\DecValTok{2}\NormalTok{,}\DataTypeTok{size =} \DecValTok{4}\NormalTok{, }\DataTypeTok{color=}\StringTok{"blue"}\NormalTok{)}\OperatorTok{+}
\KeywordTok{theme_bw}\NormalTok{(}\DataTypeTok{base_size =} \DecValTok{10}\NormalTok{) }\OperatorTok{+}\KeywordTok{theme}\NormalTok{(}\DataTypeTok{axis.text.x =} \KeywordTok{element_text}\NormalTok{(}\DataTypeTok{angle =} \DecValTok{90}\NormalTok{, }\DataTypeTok{hjust =} \DecValTok{1}\NormalTok{))}\OperatorTok{+}
\KeywordTok{labs}\NormalTok{(}\DataTypeTok{x =} \StringTok{"Duration Quantiles"}\NormalTok{, }\DataTypeTok{y =} \KeywordTok{expression}\NormalTok{(}\StringTok{"Percent Agreed to Open Account"}\NormalTok{)) }\OperatorTok{+}
\KeywordTok{labs}\NormalTok{(}\DataTypeTok{title =} \StringTok{"Longer Calls work better"}\NormalTok{)}
\end{Highlighting}
\end{Shaded}

\includegraphics{R_Markdown_Assignment__4_files/figure-latex/Query 6-1.pdf}

\subsubsection{7) Cost of Marriage!}\label{cost-of-marriage}

\paragraph{Here we see the effect of bank balance on married people
grouped by their job.The plot on the left suggests that marriage is
quite expensive for most professions, divorce even more so! The plot on
the right however, does not show any change in bank balance induced by
marital
status}\label{here-we-see-the-effect-of-bank-balance-on-married-people-grouped-by-their-job.the-plot-on-the-left-suggests-that-marriage-is-quite-expensive-for-most-professions-divorce-even-more-so-the-plot-on-the-right-however-does-not-show-any-change-in-bank-balance-induced-by-marital-status}

\begin{Shaded}
\begin{Highlighting}[]
\NormalTok{upperplot<-}\KeywordTok{ggplot}\NormalTok{()}\OperatorTok{+}\KeywordTok{geom_point}\NormalTok{(}\KeywordTok{group_by}\NormalTok{(bank,job,marital)}\OperatorTok\KeywordTok{summarise}\NormalTok{(}\DataTypeTok{avg_bal=}\KeywordTok{mean}\NormalTok{(balance))}\OperatorTok\KeywordTok{arrange}\NormalTok{(job),}\DataTypeTok{mapping=}\KeywordTok{aes}\NormalTok{(}\DataTypeTok{x=}\NormalTok{job,}\DataTypeTok{y=}\NormalTok{avg_bal,}\DataTypeTok{color=}\NormalTok{marital))}\OperatorTok{+}\KeywordTok{theme}\NormalTok{(}\DataTypeTok{axis.text.x =} \KeywordTok{element_text}\NormalTok{(}\DataTypeTok{angle =} \DecValTok{90}\NormalTok{, }\DataTypeTok{hjust =} \DecValTok{1}\NormalTok{))}\OperatorTok{+}\KeywordTok{labs}\NormalTok{(}\DataTypeTok{title =} \StringTok{"Point Plot for Average Bank Balance grouped by profession"}\NormalTok{,}\DataTypeTok{x =} \StringTok{"Job"}\NormalTok{, }\DataTypeTok{y =} \StringTok{"Avg Bank Bal"}\NormalTok{,}\DataTypeTok{caption =} \StringTok{"Bank Data"}\NormalTok{) }
\NormalTok{centerplot<-}\KeywordTok{ggplot}\NormalTok{( bank) }\OperatorTok{+}
\KeywordTok{geom_point}\NormalTok{(}\DataTypeTok{mapping =} \KeywordTok{aes}\NormalTok{(}\DataTypeTok{x =}\NormalTok{age , }\DataTypeTok{y =}\NormalTok{balance ,}\DataTypeTok{color=}\NormalTok{marital)) }\OperatorTok{+}
\KeywordTok{facet_wrap}\NormalTok{(.}\OperatorTok{~}\NormalTok{job , }\DataTypeTok{nrow =} \DecValTok{5}\NormalTok{)}\OperatorTok{+}\KeywordTok{labs}\NormalTok{(}\DataTypeTok{title =} \StringTok{"Point Plot for Average Bank Balance grouped by profession"}\NormalTok{,}\DataTypeTok{x =} \StringTok{"age"}\NormalTok{, }\DataTypeTok{y =} \StringTok{"Avg Bank Bal"}\NormalTok{,}\DataTypeTok{caption =} \StringTok{"Bank Data"}\NormalTok{) }
\KeywordTok{grid.arrange}\NormalTok{(upperplot, centerplot,}\DataTypeTok{ncol=}\DecValTok{2}\NormalTok{, }\DataTypeTok{nrow=}\DecValTok{1}\NormalTok{,}
\DataTypeTok{widths=}\KeywordTok{c}\NormalTok{(}\DecValTok{5}\NormalTok{,}\DecValTok{5}\NormalTok{), }\DataTypeTok{heights=}\KeywordTok{c}\NormalTok{(}\DecValTok{8}\NormalTok{))}
\end{Highlighting}
\end{Shaded}

\includegraphics{R_Markdown_Assignment__4_files/figure-latex/Query 7-1.pdf}

\subsubsection{8) Which age group is most likely to open a term deposit
account/ Which age group should the bank
target?}\label{which-age-group-is-most-likely-to-open-a-term-deposit-account-which-age-group-should-the-bank-target}

\paragraph{Here we see that the target group should be people over the
age bucket of 58-87 are most likely to open a term deposit accout with
the
bank}\label{here-we-see-that-the-target-group-should-be-people-over-the-age-bucket-of-58-87-are-most-likely-to-open-a-term-deposit-accout-with-the-bank}

\begin{Shaded}
\begin{Highlighting}[]
\NormalTok{df<-bank}
\NormalTok{df}\OperatorTok{$}\NormalTok{age_quantiles <-}\StringTok{ }\KeywordTok{cut}\NormalTok{(df}\OperatorTok{$}\NormalTok{age, }\KeywordTok{quantile}\NormalTok{(df}\OperatorTok{$}\NormalTok{age, }\KeywordTok{seq}\NormalTok{(}\DecValTok{0}\NormalTok{, }\DecValTok{1}\NormalTok{, }\DataTypeTok{length =} \DecValTok{15}\NormalTok{), }\DataTypeTok{na.rm =} \OtherTok{TRUE}\NormalTok{))}
\KeywordTok{ggplot}\NormalTok{(df) }\OperatorTok{+}
\StringTok{ }\KeywordTok{geom_point}\NormalTok{(}\KeywordTok{group_by}\NormalTok{(df,age_quantiles)}\OperatorTok\KeywordTok{summarise}\NormalTok{(}\DataTypeTok{percentyes=}\KeywordTok{sum}\NormalTok{((y }\OperatorTok{==}\StringTok{ "yes"}\NormalTok{) }\OperatorTok{/}\StringTok{ }\KeywordTok{n}\NormalTok{()))}\OperatorTok\NormalTok{dplyr}\OperatorTok{::}\KeywordTok{filter}\NormalTok{(age_quantiles}\OperatorTok{!=}\StringTok{"NA"}\NormalTok{)}\OperatorTok\KeywordTok{arrange}\NormalTok{(percentyes)}\OperatorTok\KeywordTok{arrange}\NormalTok{(age_quantiles),}\DataTypeTok{mapping=}\KeywordTok{aes}\NormalTok{(}\DataTypeTok{x=}\NormalTok{age_quantiles,}\DataTypeTok{y=}\NormalTok{percentyes),}\DataTypeTok{alpha =} \DecValTok{1}\OperatorTok{/}\DecValTok{2}\NormalTok{,}\DataTypeTok{size =} \DecValTok{4}\NormalTok{, }\DataTypeTok{color=}\StringTok{"blue"}\NormalTok{)}\OperatorTok{+}
\KeywordTok{theme_bw}\NormalTok{(}\DataTypeTok{base_size =} \DecValTok{10}\NormalTok{) }\OperatorTok{+}
\KeywordTok{labs}\NormalTok{(}\DataTypeTok{x =} \StringTok{"Age quantile"}\NormalTok{, }\DataTypeTok{y =} \KeywordTok{expression}\NormalTok{(}\StringTok{"Percent Agreed to Open Account"}\NormalTok{)) }\OperatorTok{+}
\KeywordTok{labs}\NormalTok{(}\DataTypeTok{title =} \StringTok{"Target Age Group for Banks"}\NormalTok{)}
\end{Highlighting}
\end{Shaded}

\includegraphics{R_Markdown_Assignment__4_files/figure-latex/Query 8-1.pdf}

\subsubsection{9) Does having a higher education make you richer, in
each profession
?}\label{does-having-a-higher-education-make-you-richer-in-each-profession}

\paragraph{We do not see many blue spots rising above the others in many
cases, we do see a few in the entrepreneur profession, but not too many.
We can say that education does not make you richer (according to the
data!). We also have a chance to see how the bank balance varies by
profession, no particular group seems to be dominating the roster here.
(note- turns out the person who spoke to the bank employee for 3000
seconds was infact unemployed, not
surprising!)}\label{we-do-not-see-many-blue-spots-rising-above-the-others-in-many-cases-we-do-see-a-few-in-the-entrepreneur-profession-but-not-too-many.-we-can-say-that-education-does-not-make-you-richer-according-to-the-data.-we-also-have-a-chance-to-see-how-the-bank-balance-varies-by-profession-no-particular-group-seems-to-be-dominating-the-roster-here.-note--turns-out-the-person-who-spoke-to-the-bank-employee-for-3000-seconds-was-infact-unemployed-not-surprising}

\begin{Shaded}
\begin{Highlighting}[]
 \KeywordTok{ggplot}\NormalTok{(bank, }\KeywordTok{aes}\NormalTok{(}\DataTypeTok{x =}\NormalTok{ duration, }\DataTypeTok{y =}\NormalTok{ balance)) }\OperatorTok{+}\StringTok{ }\KeywordTok{geom_point}\NormalTok{(}\KeywordTok{aes}\NormalTok{(}\DataTypeTok{color =}\NormalTok{ education), }\DataTypeTok{size =} \DecValTok{4}\NormalTok{, }\DataTypeTok{alpha =} \FloatTok{0.5}\NormalTok{)}\OperatorTok{+}\StringTok{ }\KeywordTok{theme_minimal}\NormalTok{() }\OperatorTok{+}\KeywordTok{facet_wrap}\NormalTok{(.}\OperatorTok{~}\NormalTok{job)}\OperatorTok{+}
\KeywordTok{labs}\NormalTok{(}\DataTypeTok{title =} \StringTok{"Point Plot for Education against Bank Balance"}\NormalTok{,}
\DataTypeTok{caption =} \StringTok{"Bank Data"}\NormalTok{) }\OperatorTok{+}
\KeywordTok{labs}\NormalTok{(}\DataTypeTok{x =} \StringTok{"Duration"}\NormalTok{, }\DataTypeTok{y =} \StringTok{"Balance "}\NormalTok{)}
\end{Highlighting}
\end{Shaded}

\includegraphics{R_Markdown_Assignment__4_files/figure-latex/Query 9-1.pdf}

\subsubsection{10) How does Education affect the job you end up
taking?}\label{how-does-education-affect-the-job-you-end-up-taking}

\paragraph{We plot two graphs, both for the same end purpose. We see
clearly that people with a higher level education end up in managerial
roles, the profession of the secondary education group seems to be
distributed uniformly, and the people with a primary level education
endup in the blue collar
roles}\label{we-plot-two-graphs-both-for-the-same-end-purpose.-we-see-clearly-that-people-with-a-higher-level-education-end-up-in-managerial-roles-the-profession-of-the-secondary-education-group-seems-to-be-distributed-uniformly-and-the-people-with-a-primary-level-education-endup-in-the-blue-collar-roles}

\begin{Shaded}
\begin{Highlighting}[]
\KeywordTok{ggplot}\NormalTok{(bank) }\OperatorTok{+}
\KeywordTok{geom_bar}\NormalTok{(}\DataTypeTok{mapping =} \KeywordTok{aes}\NormalTok{(}\DataTypeTok{x =}\NormalTok{ education, }\DataTypeTok{fill =}\NormalTok{ job), }\DataTypeTok{position =} \StringTok{"fill"}\NormalTok{)}\OperatorTok{+}\KeywordTok{labs}\NormalTok{(}\DataTypeTok{title =} \StringTok{"Bar Plot for Jobs taken based on Education"}\NormalTok{,}
\DataTypeTok{caption =} \StringTok{"Bank Data"}\NormalTok{) }
\end{Highlighting}
\end{Shaded}

\includegraphics{R_Markdown_Assignment__4_files/figure-latex/Query 10-1.pdf}

\begin{Shaded}
\begin{Highlighting}[]
\KeywordTok{ggplot}\NormalTok{(bank) }\OperatorTok{+}
\KeywordTok{geom_bar}\NormalTok{(}\DataTypeTok{mapping =} \KeywordTok{aes}\NormalTok{(}\DataTypeTok{x =}\NormalTok{ education, }\DataTypeTok{fill =}\NormalTok{ job), }\DataTypeTok{position =} \StringTok{"fill"}\NormalTok{)}\OperatorTok{+}
\StringTok{  }\KeywordTok{coord_polar}\NormalTok{()}\OperatorTok{+}\KeywordTok{labs}\NormalTok{(}\DataTypeTok{title =} \StringTok{"Polar Coordinate Plot for Jobs taken based on Education"}\NormalTok{,}
\DataTypeTok{caption =} \StringTok{"Bank Data"}\NormalTok{) }
\end{Highlighting}
\end{Shaded}

\includegraphics{R_Markdown_Assignment__4_files/figure-latex/Query 10-2.pdf}


\end{document}
